\section{Methodology}
In order to create a new application , it is necessary to follow the following steps, to create a strong foundation.
\begin{figure}[H]
\centering
\includegraphics[width=12.5cm]{img/Cicle_proj.png}
\caption[Project Development Cycle]{\footnotesize{Project development cycle}}
\label{fig:project_cycle}
\end{figure}
This diagram shows the ideal development process for a project. Since the application developed is the first version, requierements retrieval should only happen once, and the project should finish with the Release phase, once all tests have been passed correctly.\smallskip\\
For the development of the application, it has been decided to use an hybrid structure, using Object Oriented Programming (OOP) with C\# and MATLAB scripting.\smallskip\\
The main advantages of using C\# with OOP are that it allows the code to be reusable, organized and easy to maintain \cite{profile_OOP}. In this project, it has been used for the development of the Graphical User Interface and the controllers for both, web services and MATLAB script management.\smallskip\\
On the other hand, MATLAB provides a specialized environment for numerical calculations, simulations, and complex data analysis\cite{matlab}. Working directly with matrixes and arrays makes the data processing and visualitzation easier and more efficient.
