\section{List of requirements}
{This first step is crucial for the correct result. It is needed to define a clear and well defined list of requirements. This defines the main scope of the project, the different futures to be implemented and prevents from building something that is not desired. }
\smallskip
{This project, need to be able to fulfill the following requirements: }
\begin{enumerate} 
    \item {Application can be executed in CALCULA operating system (Linux)}
    \item {Download measurements from the internet}
    \begin{enumerate}
      \item {Download measurements from a defined date range}
      \item {Download measurements from a defined location}
      \item {Download measurements from the Actris-Earlinet Data Portal \cite{earlinet_web}}
      \begin{itemize}
        \item {Download ELPP products} %Una altra manera de descriure-ho¿?
        \item {Download Optical products}
      \end{itemize}
      \item {Download measurements from the Aeronet web \cite{aeronet_web}}
      \begin{itemize} %no es descarreguen tots, revisar quin eliminar
        \item {Download Raw Almucantar Sky Scan Radiance measurements}
        \item {Download Raw Hybrid Sky Scan Radiance measurements}
        \item {Download Raw Principal Plane Sky Scan Radiance measurements}
        \item {Download Raw Polarized Principal Plane Sky Scan Radiance and Degree of Polarization measurements}
        \item {Download Raw Polarized Almucantar Sky Scan Radiance and Degree of Polarization measurements}
        \item {Download Raw Polarized Hybrid Sky Scan Radiance and Degree of Polarization measurements}
      \end{itemize}
      \item 
    \end{enumerate}
    \item {Filter Earlinet measurement files depending on the the data type}
    \begin{enumerate}
      \item {002, 008}
      \item {007}
    \end{enumerate}
    \item {Execute pre-execution:}
    \begin{enumerate}
      \item {Read data of all downloaded files that are used as inputs for grasp algorithm}
      \item {Preview data in order to choose a correct value for minimum and maximum heights}
      \item {Obtain available configurations for the selected measure ID depending on available data}
      \item {Choose a configuration}
      \item {Generate a .sdat file that stores all data for selected measure ID and heights}
      \item {Create .yml configuration file containing corresponding .sdat file name and choosen configuration}
    \end{enumerate}
    \item {Execute the GRASP algorithm with corresponding configuration .yml file and .sdat file}
    \item {Give the user the possibility to plot the different results after executing the algorithm}
    \begin{itemize}
      \item {Detect all available measure IDs folders in output directory}
      \item {Generate figure data files .mat from GRASP output data in selected measure ID output directory}
      \item {Plot data from .mat files}
    \end{itemize}
    \item {Application can work with project/workspaces}
    \begin{enumerate}
      \item {User can create, open and import a project}
      \item {Downloaded data from web services are stored in the project folder}
      \item {Results of the GRASP algorithm are stored in the project folder}
    \end{enumerate}
    \item {Application can have different configurations}
    \begin{itemize}
      \item {Repository directories can be modified for both, earlinet and aeronet downloaded files}
      \item {Ouput directory can be modified}
      \item {GRASP installation directory can be modified: this path can be different for each user}
    \end{itemize}

\end{enumerate}
\bigskip