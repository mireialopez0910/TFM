\clearpage
\newpage
\section{Introduction}
Recently, the need to analyze and monitor atmospheric aerosols has grown significantly, as their negative impact has grown significantly in different sectors: human health, climatology and air transport. GRASP (Generalized Retrieval of Atmosphere and Surface Properties), which was introduced by Dubovik, is the first unified algorithm used for obtaining atmospheric properties from different remote sensing observations including satellite, ground-based and airborne measurements, both passive and active, of atmospheric radiation and their combinations\cite{grasp}.\smallskip\\
The main objective of this work is to develop a software that automates the full GRASP execution process, which includes:
\begin{enumerate}
    \item Download photometer data from the Aeronet web site
    \item Download Lidar (Light Detection and Ranging) data from the ACTRIS-EARLINET web site
    \item Combine the data downloaded from both web sites
    \item Run GRASP
    \item Plot the obtained results
\end{enumerate}
\subsection{Gantt Diagram}
The time organization of the project is shown in the Gantt diagram of \Cref{fig:gantt} representing a time lapse of one semester. It is important to note that even there are different groups of tasks for implementation and testing, tests have been executed during the implementation phase to ensure the correct functionality of the different steps.
\label{ssec:gantt}
\begin{figure}[H]
    \centering
    %\includegraphics[width=13cm]{img/diagram_gantt.png}
    \begin{ganttchart}[y unit title=0.5cm,
y unit chart=0.75cm,
vgrid,hgrid,
bar label node/.append style={text width=2.5cm,align=right},
title height=1,
today=20,
today label=Delivery,
title label font=\small,
bar label font=\footnotesize,
group label font=\small,
milestone label font=\small,
today label font=\small,
bar/.style={draw,fill=cyan!75!black},
bar incomplete/.append style={fill=orange!50},
bar height=0.5]{1}{25}

 % 2nd semester
 \gantttitle{Phases of the Project}{25} \\
 \gantttitle{2025}{17}
 \gantttitle{2026}{8} \\
 \gantttitle{September}{4}
 \gantttitle{October}{5} 
 \gantttitle{November}{4}
 \gantttitle{December}{4}
 \gantttitle{January}{4}
 \gantttitle{February}{4}\\
 
 \ganttgroup[inline=false]{Contextualization}{1}{4}\\
 \ganttbar[progress=100]{Background}{1}{3} \\
 \ganttbar[progress=100]{Requirements definition}{2}{4} \\
 \ganttgroup[inline=false]{Planning}{3}{8}\\
 \ganttbar[progress=100]{Research}{3}{5} \\
 \ganttbar[progress=100]{Matlab tests}{5}{8} \\
 \ganttbar[progress=100]{Architecture design}{4}{7} \\
 \ganttgroup[inline=false]{Implementations}{8}{19}\\
 \ganttbar[progress=100]{Download files}{8}{10} \\
 \ganttbar[progress=100]{Preview data}{10}{12} \\
 \ganttbar[progress=100]{Data combination}{12}{13.5} \\
 \ganttbar[progress=100]{Data plotting}{13.5}{14.5} \\
 \ganttbar[progress=100]{Full GRASP execution}{14.5}{15} \\
 \ganttbar[progress=100]{Tests fixing}{15}{19} \\
 \ganttgroup[inline=false]{Testing}{15}{19}\\
 \ganttbar[progress=100]{1st version tests}{15}{17} \\
 \ganttbar[progress=100]{Final version tests}{17}{18.5} \\
 \ganttgroup[inline=false]{Documentation}{8}{20}\\
 \ganttbar[progress=100]{Continuous documentation}{8}{20} \\
 \ganttgroup[inline=false]{Review}{4}{20}\\
 \ganttbar[progress=100]{Continuous review}{4}{20} \\
 \ganttgroup[inline=false]{Finalization}{20}{23}\\
 \ganttbar[progress=25]{Presentation}{20}{23}
\end{ganttchart}

    \caption[Project's Gantt diagram]{\footnotesize{Gantt diagram of the project}}
    \label{fig:gantt}
\end{figure}
\subsection{Deviation from the initial plan}
\bigskip